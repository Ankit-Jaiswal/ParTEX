\documentclass[master.tex]{subfiles}
\setcounter{section}{1}

\newcommand{\poly}[3]{\sum_{#1=0}^#3 {#2}_{#1}x^{#1}}
\newcommand{\polyex}[2]{{#1}_#2 x^{#2} + {#1}_{#2-1}x^{#2-1} + \ldots + {#1}_1 x + {#1}_0}

\begin{document}
\section{Ring Theory}
\subsection{Basic Definitions}
\newcommand{\F}{\mathbb{F}} \newtheorem*{notation}{Notation}
% Lecture 10/27/2016

\begin{defn*}
  A \emph{ring} is set \(R\) together with two binary operations \(+\) and \(\cdot\) satisfying
  \begin{enumerate}[label=(\roman*)]
  \item \((R,+)\) is an abelian group (denote the additive identity by \(0\))
  \item \(\cdot\) is associative, \((xy)z=x(yz)\) for all \(x,y \in R\)
  \item There exists a multiplicative identity (denoted \(1 \in R\)).
  \item Distribution laws hold:
    \begin{align*}
      x(y+z) &= xy + xz\\
      (x+y)z &= xz+yz
    \end{align*}
  \end{enumerate}
\end{defn*}

\begin{defn*}
  A \emph{commutative ring} is a ring with the additional property that \(xy=yx\) for all \(x,y \in R\).
\end{defn*}
Notice that if multiplication is commutative either distribution law implies the other. One of the most basic
observations one can make is that \(R\) is the trivial ring \(\iff 1=0\).

\begin{example*}
  A few familiar rings: \[\Z,\Z_n,\Q,\R,\C\] A non-commutative ring:
  \[M_n(\F), n \ge 2\] or more generally take any ring \(R\).
\end{example*}

\begin{defn*}
  A \emph{skew field (or division ring)} is a ring \(R\) such that \(R \neq \{0\}\) and \(R \setminus \{0\}\) is a group
  under multiplication.
\end{defn*}

\begin{defn*}
  A \emph{field} is a commutative skew field.
\end{defn*}

\begin{example*}
  An example of a skew field is
  \[\H = \R + \R i + \R j + \R k\]
\end{example*}

One may form direct products of rings in the usual manner.

\begin{defn*}
  A \emph{ring homomorphism} is map \(\varphi \colon R \to S\) between rings \(R\) and \(S\) satisfying
  \begin{enumerate}[label=(\arabic*)]
  \item \(\varphi(x+y)=\varphi(x)+\varphi(y)\)
  \item \(\varphi(xy)=\varphi(x)\varphi(y)\)
  \item \(\varphi(1_R)=1_S\)
  \end{enumerate}
\end{defn*}

In keeping with our demand that all of rings have unity, we also force our homomorphisms to respect the unital
structure. In particular our ring homomorphisms are morphisms in the category of commutative rings.

\begin{example*}
  For us a map \funcdeclaration{\varphi}{R}{R \times R}{r}{(r,0)} is not a ring homomorphism as
  \(1_{R \times R} = (1,1)\) while \(\varphi(1)=(1,0)\).
\end{example*}

\begin{defn*}
  A subset \(S\) of a ring \(R\) is a \emph{subring} of \(R\) if
  \begin{enumerate}[label=(\arabic*)]
  \item \(S\) is closed under both operations
  \item \(1_R \in S\)
  \end{enumerate}
\end{defn*}

\begin{example*}
  Under this definition \(R \times \{0\}\) is not a subring of \(R \times R\) as \(1_R \not \in R \times R\) as long as
  \(R\neq\{0\}\).
\end{example*}

\begin{defn*}
  An element \(x \in R\) is said to be a \emph{zero divisor} of \(R\) if there exists an element \(y \in R\) such that
  \(y \neq 0\) and
  \[xy = 0 \qquad \text{ or } \qquad yx = 0.\]
\end{defn*}

\begin{defn*}
  The commutative ring \(R\) is an \emph{integral domain} if its only zero divisor is \(0\).
\end{defn*}

\begin{example*}
  \(\Z_n\) integral domain \(\iff n\) is prime
\end{example*}

\begin{prop*}
  \(R\) finite integral domain \(\implies\) \(R\) is field.
\end{prop*}

\begin{defn*}
  A \(x \in R\) is a \emph{unit} if \(xy=1\) and \(yx=1\).
\end{defn*}

\begin{defn*}
  The group of units \(R^\times\) of a ring is
  \[R^\times = \{x \in R \mid x\text{ is a unit}\}.\]
\end{defn*}

\begin{example*}
  \begin{align*}
    \Z^\times &= \{-1,1\}\\
    \Z_n^\times &= \{\bar{a} \mid (a,n)=1\}
  \end{align*}
\end{example*}

\begin{prop*}
  If \(\varphi \colon R \to S\) is a homomorphism and \(x \in R^{\times}\) then \(\varphi(x) \in S^\times\). In
  particular this means \(\varphi(R^\times) \le S^\times\).
\end{prop*}

Even if \(\varphi \colon R \twoheadrightarrow S\) the image \(\varphi(R^\times)\) might not equal \(S^\times\).

\begin{example*}
  Consider the surjective function \funcdeclaration{\varphi}{\Z}{\Z_n}{a}{\bar{a}} but \(|\Z_n^\times|=\varphi(n)>2\) if
  \(n \ge 7\)
\end{example*}

\begin{notation}
  From this point onwards \(R\) denotes a \textbf{commutative ring}.
\end{notation}

\begin{defn*}
  A non-empty subset \(I \subset R\) is an \emph{ideal} if
  \begin{enumerate}[label=(\roman*)]
  \item \(x,y \in I \implies x+y \in I \)
  \item \(x \in I, r \in R \implies rx \in I\)
  \end{enumerate}
\end{defn*}
Condition one may be rephrased as \((I,+)\) forms an abelian group.
\begin{prop*}
  An ideal \(I=R \iff 1 \in R \iff I \cap R^\times \neq \emptyset\)
\end{prop*}

\begin{notation}
  If \(I\) is an ideal of \(R\) we denote it as \(I \unlhd R\).
\end{notation}

\begin{prop}
  A ring homomorphism \(\varphi \colon R \to S\) if \(I \unlhd R\), \(J \unlhd S\) then
  \[\varphi(I) \unlhd S \iff \varphi \text{ is surjective }\]
  while
  \[\varphi^{-1} \unlhd R \text{ is always true.}\]
\end{prop}

\begin{defn*}
  Let \(I,J \unlhd R\) then we define
  \begin{align*}
    I+J &:= \{x+y \mid x \in I, y \in J\} \unlhd R\\
    IJ  &:= \left\{\sum_{l=0}^nx_l y_l \mid n \in \N, x_l \in I, y_l \in J\right\}.
  \end{align*}
\end{defn*}
We have defined \(IJ\) in the above manner to force \(IJ\) to be an abelian group, and hence an ideal.

\begin{defn*}
  Given an ideal \(I \unlhd R\) we may form the \emph{quotient ring} denote \(R/I\) as follows
  \[R/I = \{a+I\mid a \in R\}.\] This forms a ring under addition and multiplication by representatives.
\end{defn*}

\begin{thm}[Isomorphism Theorems For Rings] Given a ring homomorphism \(\varphi \colon R \to S\). Then
  \begin{enumerate}[label=(\alph*)]
  \item \(\varphi(R)\) is a subring of \(S\) and \(\ker(\varphi)=\{r \in R \mid \varphi(r)=0\} \unlhd R\)
  \item \(R/\ker \varphi \isom \varphi(R)\)
  \item \(I,J \unlhd R\) with \(I \subset J \implies\)
    \[J/I\unlhd R/I\] and
    \[(R/I)/(J/I) \isom R/S.\]
  \end{enumerate}
\end{thm}

\begin{prop}{Correspondence Theorem}
  Let \(I \unlhd R\) define \funcdeclaration{\pi}{R}{R/I}{a}{\bar{a}} Then there is a bijection
  \[\{J \unlhd R \mid I \subset J\} \longleftrightarrow \{\text{Ideals of } R/I \}\] given by
  \[J \longmapsto J/I\] and
  \[L \longmapsto \pi^{-1}(L).\]
\end{prop}

\begin{proof}
  First \(\bar{J}=J/I=\pi(J)\unlhd \bar{R}\) since \(\pi\) is surjective \(\pi^{-1}(L) \unlhd R\) and contains
  \(I=\pi^{-1}(O)\). Claim: \(gf(J)=\pi^{-1}(J/I)=J\). Assume \(x \in R\) with \(x \in \pi^{-1}(J/I)\) then
  \(\pi(x) \in J/I\). Then \(x+I=y+I\) for some \(y \in J\). Thus \(x \in y + I \subset J\). Lastly
  \(fg(L)=\pi(\pi^{-1}(L))\) since \(\pi\) is surjective.
\end{proof}

\begin{defn}
  A proper ideal \(I \lhd R\) is called
  \begin{enumerate}
  \item a \emph{prime ideal} if
    \[xy \in I \implies x \in I \text{ or } y \in I.\]
  \item a \emph{maximal ideal} if
    \[\forall J \lhd R,\ I \subseteq J \implies I=J.\]
  \end{enumerate}
\end{defn}

\begin{prop}
  For every \(I \lhd R\) there exists a maximal ideal \(M\) such that \(I \subseteq M\) and \(M \lhd R\).
\end{prop}

\begin{proof}[Short Proof]
  This follows from Zorn's Lemma.
\end{proof}

Here we will be pedantic and write out the proof in its entirety as it sets the flavor of proof for the existence of
other maximal entities.

\begin{lem*}[Zorn's Lemma]
  If \(A\) is a nonempty partially ordered set such that every chain in \(A\) has an upper bound in \(A\), then \(A\)
  contains a maximal element.
\end{lem*}

\begin{proof}[Detailed Proof]
  Define set \(S=\{J \lhd R \mid I \subset J\} \neq \emptyset\). This set is never empty as \(I \in S\) by
  construction. The set \(S\) is partially ordered \(j \le J' \iff J \subseteq J'\). Now check that every chain in \(S\)
  has an upper bound. Let \((I_\alpha)_{\alpha in A}\) be a chain in \(S\) where \(\alpha, \beta \in A\) then
  \(J_\alpha \subseteq J_\beta\) or \(J_\beta \subseteq J_\alpha\). Here we can compare any two elements. The candidate
  for an upper bound is
  \[J=\bigcup_{\alpha \in A} J_\alpha.\] We must check the following requirements:
  \begin{enumerate}[label=(\arabic*)]
  \item \(I \subset J\)
  \item \(J \unlhd R\)
  \item \(J \neq R\).
  \end{enumerate}
  The condition (1) is immediate as \(I \subset J_\alpha\) for each \(\alpha \in A\). To see (2) let \(x,y \in J\) and
  \(r \in R\). Then there exists \(\alpha, \beta \in A\) such for \(x \in I\) and \(y \in J_\beta\) without loss of
  generality \(J_\alpha \subseteq J_\beta\). Then \(x+y \in J_\beta \subseteq J\). Since \(J_\alpha\) is an ideal for
  \(x \in J_\alpha\) we have \(rx \in J_\alpha \subset J\). Thus \(J\) is an ideal of \(R\). Lastly to see (3) notice
  \(1 \not \in J\) since \(1 \not \in J_\alpha\) for all \(\alpha \in A\) as each of these ideals is proper. Recall an
  ideal is the whole ring if and only if it contains unity, thus we have \(J \neq R\). We have shown that the conditions
  of Zorn's Lemma hold, hence \(S\) has a maximal element.
\end{proof}

\begin{prop}
  \(I \unlhd R\)
  \begin{enumerate}[label=(\alph*)]
  \item \(I\) is a prime ideal \(\iff\) \(R/I\) integral domain.
  \item \(I\) is a maximal ideal \(\iff\) \(R/I\) is a field.
  \end{enumerate}
\end{prop}

\begin{proof}
  \begin{enumerate}[label=(\alph*)]
  \item For \(x,y \in R\), the following are equivalent:
    \[xy \in I \implies x \in \text{ or } y \in I\\\] \[\iff\]
    \[\bar{x}\bar{y}=\bar{0} \implies \bar{x}=\bar{0} \text{ or } \bar{y}=\bar{0}\] \[\iff\] \[\text{R/I is an
        integral domain}\]
  \item \(I\) is maximal \(\iff I + Rx = \{a+rx \mid r \in R\} \unlhd R\)
    \[\iff\]
    \[\bar{R}\bar{x}=\bar{R} \ \forall \bar{x} \neq \bar{0} \in \bar{R}\]
    \[\iff\]
    \[\text{All non-zero elements of \(\bar{R}\) are units.}\]
  \end{enumerate}

  \begin{cor*}
    Every maximal ideal is prime.
  \end{cor*}
\end{proof}

\begin{rmk}
  Recall from the Correspondence Theorem that there is a bijection between ideals of \(R\) that contain an ideal \(I\)
  and the ideals of \(R/I\). This bijection may be restricted to either the prime ideals or the maximal ideals. Yielding
  a bijection between the prime/maximal ideals of \(R\) containing \(I\) and the prime/maximal ideals of \(R/I\).
\end{rmk}

\begin{defn*}
  Given \(S \subseteq R\) the \emph{ideal generated by \(S\) in \(R\)} denoted \((S)\) is the smallest ideal of \(R\)
  containing \(S\). More specifically it is defined to be
  \[(S)=\bigcap_{S \subset I \unlhd R} I\]
\end{defn*}

\begin{notation}
  Given a finite set \(S=\{x_1,\ldots,x_n\} \subset R\) we write \((x_1,\ldots,x_n)\) for
  \(\left(\{x_1,\ldots,x_n\}\right)\) which is exactly the set of all linear combinations of the elements of \(S\) with
  coefficients from \(R\).
\end{notation}

\begin{defn*}
  For \(x \in R\) the ideal \((x)=Rx=\{rx \mid r \in R\}\) is called the principal ideal generated by the element \(x\).
\end{defn*}

\begin{defn}
  An integral domain \(R\) is called a \emph{principal ideal domain (PID)} if every ideal of \(R\) is principle.
\end{defn}

\begin{examples}
  \(Z\), \(\F\), \(\F[x]\)
\end{examples}

\begin{notation}
  \(a,b \in R\) we say \emph{\(a\) divides \(b\)} or \(b\)is a multiple of \(R\) if \(\exists c \in R\) such that
  \(ac=b\) denoted \(a \mid b\).
\end{notation}

\begin{rmk} Let \(a \in R\)
  \begin{enumerate}[label=(\alph*)]
  \item \(\{b \in R \mid a \mid b\}=(a)\)
  \item \(a \mid 0\) since \(0=a \cdot 0 \)
  \item \(u \in R^\times \implies u \mid a\) since \(u(u^{-1}a)=a\)
  \item \(ac=b\) should not be written as \(c=\frac{b}{a}\)since in general \(\frac{b}{a}\) is ambiguous. Ex in \(\Z_6\)
    \(2 \times 2 = 4\) but \(2 \times 5 = 4\) also so is \(\frac{4}{2}=2\) or 5.
  \end{enumerate}
\end{rmk}

\begin{defn}
  Two ideals \(I, J \unlhd R\) are called \emph{comaximal} if \(I+J=R\) (\(\iff \exists a \in I, b \in J\) such that
  \(a+b=1\))
\end{defn}

\begin{lem}
  Given finitely many ideals \(I_1,\ldots, I_n\) that are pairwise comaximal, then \(I_1 \cdots I_{n-1}\) and \(I_n\)
  are comaximal and \(\bigcap_{j=1}^n I_J = I_1 \cdots I_n\)
\end{lem}

\begin{proof}
  We proceed by induction on \(n\). Consider when \(n=2\). One must show that that \(I_1,I_2\) comaximal
  \(\implies I_1 \cap I_2 \subset I_1I_2\). The other inclusion is obvious. If \(I_1\) and \(I_2\) are comaximal then
  there exists \(a \in I_1\) and \(b \in I_2\) such that \(a+b=1\). Take \(x \in I_1 \cap I_2\) then \(x=x \cdot 1 =
  x(a+b)=xa+xb\). Since \(xa,xb \in I_1 I_2\) we have that \(x \in I_1 I_2\) as desired.

  Now we consider the case when \(n \ge 3\). If \(I_j,I_n\) are comaximal for \(1 \le j \le n-1\) then there must exist
  \(a_j \in I_j\) and \(b_j \in I_n\) with \(a_j+b_j = 1\). Now
  \[1=1 \cdots 1 = \prod_{j=1}^{n=1}{a_j + b_j}\]
  Expanding and we see
  \[\prod_{j=1}^{n-1}(a_j+b_j) \in a_1 \cdots a_{N=1}+I_n\]
  Then \((I_1 \ldots I_{n-1}I_n)=I_1 \ldots I_{n-1} \cap I_n\)
  which by the inductive hypothesis is exactly \(\bigcap_{j=1}^n I_J\).
\end{proof}

\begin{prop}[Chinese Remainder Theorem]
  Let \(I_1, \ldots, I_n \unlhd R\) and
  \funcdeclaration{\varphi}{R}{\bigtimes_{j=1}^n}{r}{(r+I_j)_{1 \le j \le n}}
  \begin{enumerate}[label=(\alph*)]
  \item \(\varphi\) is a ring homomorphism with \(\ker \varphi = \bigcap I_j\)
  \item If \(I_1, \ldots I_n\) are pairwise comaximal then \(\ker \varphi = \prod_{j=1}^n I_n\) and if \(\varphi\) is
    surjective then
    \[\sfrac{\varphi}{ \prod_{j=1}^n I_j} \cong \bigtimes_{j=1}^n R/I_j\]
  \end{enumerate}
\end{prop}

\begin{proof}
  \begin{enumerate}[label=(\alph*)]
  \item trivial
  \item \(\ker \varphi = \prod_{j=1}^n I_j\) follows (a) and 2.1.12.
    By 2.1.12 we get \(\forall 1 \le l \le n\) that the ideals \(\hat{I_l}=\prod_{j\neq l, j=1} I_j\) and \(I_l\) are
    comaximal. Then there exists \(x_l \in \hat{I_l}\)and \(y_l \in I_l\) such that \(x_l + y_l = 1\). Then \(x_l \equiv
    0 \pmod{I_j \forall j \neq l}\) (this means that \(\bar{x_l} \equiv 1 \pmod{I_l}\) (or \(x_l + I_l = 1 I_l\))).

    Given \((a_j + I_j)_{1 \le j \le n} \in \bigtimes_{j=1}^n{R/I_j}, (a_j \in R)\). We set \(x=sum_{l=1}^n{a_l x_l} \in
    R\). Now let \(\bar{\phantom{}}\) denote mod \(I_j\) for some fixed \(j\).

    \begin{align*}
      \bar{x} &= \sum_{l=1}^n \bar{a_l}\bar{x_l}\\ \tag{\(\bar{x_l}=\bar{0}\) if \(\l \neq j\)}
              &= \bar{a_j}\bar{x_j}
              &= \bar{a_j}\bar{1}\\
              &= a_j + I_j.
    \end{align*}
    This argument may be made for each \(1 \le j \le n\). This means that \(\varphi(x)=(\bar(a_j)_{1 \le j \le n})\). We
    have thus shown our map to be surjective.
  \end{enumerate}
\end{proof}
  \begin{example*}[Special Case \(R = \Z\)]
    \begin{enumerate}[label=(\alph*)]
    \item \(n=2\), \(I_1=(m)\), \(I_2=(n)\) with \((m,n)=1 \implies\)
      \[\sfrac{\Z}{mn} \cong \sfrac{\Z}{m} \times \sfrac{\Z}{n}\]
      See 1.2.3 where we already derived this fact using more elementary means.
    \item \(n\) arbitrary \(p_1 \ldots p_n\) are distinct primes, \(l_1, \ldots, l_n \in \N\), \(I_j = (p_j^{e_j})_{1
        \le j \le n} \implies\) \[\sfrac{\Z}{\primedecomposition{p}{e}{n}} \cong \bigtimes_{j=1}^n
        \sfrac{\Z}{p_j^{e_j}}.\]
      This means there is always a solution to the system of congruences \(x \equiv a_j \pmod{p_j^{ej}}\). The injection
      is uniquely determined modulo the product.
    \end{enumerate}
  \end{example*}
  Notice that the above follows from the fact \(a,b \in R \implies (a)(b)=(ab)\).

  \subsection{Ring of Fractions}
  One may ask if there is a canonical way to embed every ring \(R\) into some larger field. If \(R\) is not an integral
  domain it can not be embedded into a field, because then the field would contain zero divisors. However the answer is
  always yes for integral domains.  We generalize the construction of \(\Q\) from \(\Z.\) Consider the embedding
  \[\Z \hookrightarrow \Q.\]
  The elements of \(\Q\) can be viewed as equivalence classes of \(\Z \times Z \setminus \{0\}\) under the relation
  \[(a,b) \sim (c,d) \iff ad=bc.\]

  Main motivations for introducing the rings of fractions:
  \begin{enumerate}
  \item Embedding integral domains into fields
  \item The concept of localization (a way to add inverses to a ring)
  \end{enumerate}

  Compare with [DF] sections 7.5 and 15.4.

  \begin{defn}
    A subset \(D\) of \(R\) is called \emph{multiplicatively closed} if
    \begin{enumerate}[label=(\arabic*)]
    \item \(d,d' \in D \implies dd' \in D\)
    \item \(1_R \in D\)
    \end{enumerate}
  \end{defn}

  \begin{example*}
    \begin{enumerate}[label=(\alph*)]
    \item \(R\) is integral domain \(\implies R \setminus \{0\}\) is multiplicatively closed. (This is in fact a
      characterization of integral domains)
    \item Any ring \(R\) with prime ideal \(P \lhd R \leadsto D = R \setminus P\). Notice that (a) is a consequence of
      this. Since \(\{0\}\) is prime \(\iff R\) is an integral domain.
    \item \(x \in R \leadsto D= \{x^n \mid n \in \N_0\}\)
    \end{enumerate}
  \end{example*}

  The goal is to invert the elements of \(D\) inside some larger ring. We will denote the ring \(D^{-1}R\). Following
  the construction of \(\Q\) from \(\Z\) one would think that equivalence relation \((r,d)\sim (r',d')\) would do the
  trick, however this relation fails to be transitive for \(R\) that are not integral domains (one would be required to
  cancel at a certain step). The correct equivalence relation is given by
  \[(r,d) \sim (r',d') \iff \exists e \in D \text{ s.t. } e(rd'-r'd)=0. \] It is tedious to check that this is
  transitive, however it is. Notice that in the case that \(R\) is an integral domain and \(0 \neq \in D\) this reduces
  to the naive relation as \(e(rd'-r'd)=0\) implies \(rd'-r'd=0\) since there are no zero divisors and \(e\) by
  assumption is nonzero.

  \begin{notation}
    In this context \(\frac{r}{d}\) denotes the equivalence class with representative \((r,d)\).
  \end{notation}

  Now we define
  \[D^{-1}R= \left\{\frac{r}{d} \mid r \in R, d \in D\right\}.\] We endow it with the operations
  \begin{align*}
    \frac{r}{d}+\frac{s}{e} &\coloneqq \frac{re+sd}{de}\\
    \frac{r}{d}\cdot\frac{s}{e} &\coloneqq \frac{rs}{de}.
  \end{align*}
  Since we are working with equivalence classes one must check the well-definedness of these operations. One may do this
  by using the definition of the equivalence relation. Finally one may check that \((D^{-1}R,+,\cdot)\) forms a
  commutative ring with unity \(1=\frac{1_R}{1_R}\). Further we have accomplished are goal as each \(d \in D\) is a unit
  in \(D^{-1}R\) with inverse \(\frac{1}{d}\).

  \begin{rmk*}
    \[D^{-1}R = \{0\} \iff 0 \in D\\\]
    \[D^{-1}R = \{0\} \iff \frac{1}{1}=\frac{0}{1} \iff \exists d \in D \text{ s.t. } d(1-0)=0 \iff d=0\]
  \end{rmk*}
  Dummit and Foote over restrictive by excluding all zero divisors from being included in \(D\). Rather one must exclude
  both a zero divisor and its partner from being in \(D\) as then \(D\) would contain zero, however there is nothing
  wrong with having one of the pair. However one must exclude nilpotent elements as we saw that
  \(\{x^n \mid n \in \N_0\}\) is multiplicatively closed meaning \(D\) would also contain zero.
  \begin{example*}
    An important example of \(D\) containing zero divisors, but still being non-trivial is
    \[D=R \setminus P\] where \(P\) is a prime ideal of \(R\). The ideal ideal \(P\) must contain one of each pair of
    the zero divisors of \(R\) otherwise \(R/P\) would not be an integral domain, but \(R/P\) is an integral domain
    since \(P\) is a prime ideal.
  \end{example*}
  We have a canonical map: \funcdeclaration{j}{R}{D^{-1}R}{r}{\frac{r}{1}} If \(R\) is not an integral domain then \(j\)
  is not injective (even if \(0 \not \in D\)). One may see this by realizing
  \[\ker j = \{r \in R \mid dr = 0\}.\]
  We summarize our construction and its properties in the following proposition:
  \begin{prop}Given a commutative ring \(R\) with unity.
    \begin{enumerate}[label=(\alph*)]
    \item For any multiplicatively closed subset \(D \subseteq R\), \(\left(D^{-1}R,+,\cdot\right)\) is a commutative ring with unity
      called the ring of fractions with respect to \(D\).
    \item \(D^{-1}R=\{0\}\iff 0 \in D\)
    \item There is a canonical map: \funcdeclaration{j}{R}{D^{-1}R}{r}{\frac{r}{1}} under which
      \(j(D) \subseteq D^{-1}R^\times\).
    \item If \(R\) is an integral domain and \(0 \not \in D\) then \(j\) is injective.
    \item If \(R\) is an integral domain and \(D=R \setminus \{0\}\) then \(D^{-1}R\) is a field containing a copy of
      \(R\), namely \(j(R)\). In this case \(D^{-1}R\) is called the field of fractions of \(R\).
    \end{enumerate}
    \begin{notation}
      \[\FieldFrac{R}=(R\setminus\{0\})^{-1}R.\]
    \end{notation}
  \end{prop}
  Notice that in the above definition we referred to the resultant construction with the article ``the''. This hints at
  some uniqueness floating around. We make this rigorous by characterizing the universal property of our construction.
  \begin{prop}[Universal Property of \(D^{-1}R\)]
    For any ring \(S\) and any ring homomorphism \(\varphi \colon R \to S\) which satisfies
    \(\varphi(D) \subset S^\times\), then there exists a unique ring homomorphism
    \(\tilde{\varphi} \colon D^{-1}R \to S\) making the following diagram commute
    \begin{figure}[h]
      \centering
      \begin{tikzcd}
        R \arrow[r,"\varphi"] \arrow[d,"j"]& S\\
        D^{-1}R \arrow[ru,dashed,"\tilde{\varphi}"]
      \end{tikzcd}
    \end{figure}

    Note that we required $\phi(D) \subset S^\times$ because each $d \in D$ is invertible in $D^{-1}R$, and we need it to remain invertible in $S$.
  \end{prop}
  \begin{prop}
    Let \(R\) be an integral domain with field of fractions \(F=D^{-1}R\). If \(\varphi \colon R \to K\) is an injective
    ring homomorphism, where \(K\) is a field, then \(K\) contains an isomorphic copy of \(F\).
  \end{prop}
  In this sense \(F\) is the ``smallest'' field that contains \(R\).
  \begin{proof}
    The injectivity of \(\varphi\) implies that \(\varphi(D) \subset K \setminus \{0\} = K^\times\). Thus we may invoke
    the universal property of the ring of fractions, hence there exists \(\tilde{\varphi} \colon F \to K\) with
    \(\tilde{\varphi} \circ j = \varphi\). This map is not the trivial map.  \(\tilde{\varphi}\) is injective because $\varphi$ is injective and \(\tilde{\varphi} \circ j = \varphi\).  So we have our embedding.
  \end{proof}
  Now we examine the ideal structure of \(D^{-1}R\) where \(R \neq 0\) and \(0 \not \in D\).
  \begin{rmk} \mbox{}
    \begin{enumerate}[label=(\alph*)]
    \item \[I \unlhd R \leadsto D^{-1}I = \left\{\frac{a}{d} \mid a \in I, d \in D\right\}\]
    \item \[D^{-1}I = D^{-1}R \iff D \cap I \neq \emptyset\]
    \end{enumerate}
  \end{rmk}

  \begin{prop}
    Define the map
    \begin{align*}
      \{I \lhd R \mid D \cap I = \emptyset\} &\longrightarrow \{\text{proper ideals of } D^{-1}R\}\\
      I &\longmapsto D^{-1} I.
    \end{align*}
    \begin{enumerate}[label=(\alph*)]
    \item The map is surjective. In general it is not injective.
    \item If \(P \lhd R\) is a prime ideal of \(R\) and \(P \cap D = \emptyset\), then \(j(D^{-1}P)=P\).
    \item The map restricted to the prime ideals of \(R\) is a bijection.
    \end{enumerate}
  \end{prop}

  \begin{defn}
    Observe that if \(P \lhd R\) is a prime ideal, then \(D = R \setminus P\) is multiplicatively closed (To prove this, use the fact that a prime ideal is generated by a prime element).  In this case, the ring of fractions \(D^{-1}R\)
    is denoted \(R_P\) and called the \emph{localization} of \(R\) at \(P\).
  \end{defn}

  \begin{cor}
    The map
    \begin{align*}
      \{Q \lhd R \mid Q \text{ prime and } Q \subset P\} &\longrightarrow \{\text{prime ideals of } R_P \}\\
      Q& \longmapsto D^{-1}Q
    \end{align*}
    is an inclusion preserving bijection.
  \end{cor}
  Consequence \(D^{-1}P=PR_P\) is the unique maximal ideal of \(R_P\).

  \begin{defn*}
    A commutative ring is called a \emph{local ring} if it has a unique maximal ideal.
  \end{defn*}
  \begin{example}
    \(R=\Z, P=(p)\). Then
    \[\Z_{(P)}=\left\{\frac{a}{b} \mid a,b \in \Z, p \;\text{does not divide}\; b\right\} \subseteq \Q = \Z_{(0)}\]
    By the previous result, \(\Z_{(P)}\) has two ideals: \(\{0\}\) and \((p)\Z_{(p)}\). Notice that
    \[(p)\Z_{(p)}=\left\{\frac{a}{b} \mid p \text{ does not divide } b, p \mid a\right\}\]
  \end{example}
  \subsection{Polynomial Rings}
  How \textbf{not} to define the polynomial ring \(R[x]\).
  \begin{align*}
    R[x] = \left\{f \colon R \to R \mid \exists a_{i} \in R : f(x)=\sum_{i=1}^na_ix^i\right\}
  \end{align*}
  i.e. as functions that admit a representation as a polynomial.
  \begin{example}
    \(R = \Z_p = \F_p\) consider \(x, x^p \in \F_p[x]\). Problem: \(x\) and \(x^p\) define the same function but have
    different representations.
  \end{example}

  \begin{defn}
    The \emph{polynomial ring} (in one variable over \(R\)) \((R[X],+,\cdot)\) is the set
    \begin{align*}
      R[X] \coloneqq \left\{(a_i)_{i \in \N_0} \mid a_i \in R, \text{ with all but finitely many entries zero}\right\}
    \end{align*}
    with operations
    \begin{align*}
      (a_i) + (b_i) &\coloneqq (a_i + b_i)\\
      (a_i)(b_i) &\coloneqq \left(\sum_{i=0}^k a_ib_{k-i}\right)_{k \in \N_0}.
    \end{align*}
  \end{defn}
  One may form this checks a commutative ring with identity \(1=(1,0,0,\ldots)\). What requires the most work is
  checking that the multiplication is associative.
  \begin{rmk*}
    Elements of \(R[x]\) are called polynomials. There is a natural embedding
    \begin{align*}
      R &\hookrightarrow R[X]\\
      a &\mapsto (a,0,0,\ldots)
    \end{align*}.
    There is a distinguished element called \(x\) in \(R[X]\), namely we let
    \[x = (0,1,0,\ldots).\] For \(a=(a_i)_{i \in \N_0}\) one checks
    \begin{align*}
      (xa)_0 &= 0\\
      (xa)_i &= (a_{i-1})
    \end{align*}.
    In other words this element \(x\) has the property that it shifts the coefficients.
  \end{rmk*}

\begin{defn}
  Any element \(0 \neq a \in R[x]\) can be uniquely written in the form \(\poly{i}{a}{d}\) with all \(a_i \in R\) and
  \(a_d \neq 0\). The integer \(d\) is called the \emph{degree} and is denoted \(\deg(a)\). Note that
  \[deg(a) = 0 \iff a \in R \setminus \{0\}\] The element \(a_d\) is called \emph{the leading coefficient} of \(a\) and
  is denoted by \(\ell(a)\). The polynomial \(a\) is \emph{monic} if \(\ell(a)=1\). We also define
  \(\ell(0) \coloneqq 0\) and \(\deg(0) \coloneqq - \infty\). This representation admits the operations in the follow
  form:
  \begin{align*}
    \poly{i}{a}{d}+\poly{i}{b}{e} &= \sum_{i=0}^{\max\{d,e\}}(a_i+b_i)x^i\\
    \poly{i}{a}{d}\poly{j}{b}{e}  &= \sum_{k=0}^{d+e}\left(\sum_{i=0}^k a_i b_{k-i}\right)x^k
  \end{align*}
  where \(a_i \coloneqq 0\) for \(i > d\) and \(b_j \coloneqq 0\) for \(j > e\). In expanded form:
  \begin{align*}
    \poly{i}{a}{d}\poly{j}{b}{e} &= a_0 b_0 + (a_0 b_1 + a_1 b_0)x + \ldots + a_db_e x^{d+e}
  \end{align*}
\end{defn}
I have omitted a very tedious verification of the uniqueness of the representation.
\begin{rmk}
  In a similar way we define the \emph{power series ring} denoted \(R[[x]]\) as follow
  \[R[[x]] \coloneqq \left\{(a_i)_{i \in \N_0}\} \mid a_i \in R\right\}.\] Here we have simply dropped the requirement
  that all but a finite amount of terms must be zero. We define the operations in the same manner. Further notice that
  there are no infinite sums being carried out. Also there is a natural inclusion
  \[R[x] \hookrightarrow R[[x]]\]
\end{rmk}

\begin{lem}
  Let \(f,g \in R[x]\).
  \begin{enumerate}[label=(\alph*)]
  \item \(\deg(f+g) \le \max \{\deg f, \deg g\}\)
  \item If \(\ell(f)\) is not a zero divisor of \(R\) we have
    \[\deg(fg) = \deg(f)+\deg(g)\]
  \end{enumerate}
\end{lem}


\begin{example*}
  If \(l(f)\) is a zero divisor we might have \(\deg f \ deg g < \deg f + \deg g\). Consider \(R = \Z_6\) then
  \begin{align*}
    (1+2x)(1+3x) = (1+5x)
  \end{align*}
\end{example*}

\begin{prop}
  Let \(R\) be an integral domain. Then
  \begin{enumerate}[label=(\alph*)]
  \item \(f,g \in R[x]\), \[\deg(fg)=\deg(f)+deg(g)\]
  \item \(R[x]\) is an integral domain
  \item \(R[x]^\times = R^\times\)
  \end{enumerate}
\end{prop}

\begin{proof}
  \begin{enumerate}[label=(\alph*)]
  \item Assume \(f \neq 0\) and \(g \neq 0\). Then since \(R\) is an integral domain \(\ell(f)\) and \(\ell(g)\) are not
    zero divisors. Then by lemma 2.3.4b we have that \(\deg(fg)=\deg(f)+\deg(g)\).
  \item \(f \neq 0 \neq g \implies \deg(fg)=\deg(f)+\deg(g)\) thus \(\deg(fg) \ge 0\), hence \(fg \neq 0\)
  \item \(R \hookrightarrow R[x] \implies R^\times \subseteq R[x]^\times\). Suppose \(f,g\) are units in \(R[x]\), and
    then use the previous part to show \(\deg(f)+\deg(g)=0\), and hence \(f,g \in R\).
  \end{enumerate}
\end{proof}

\begin{rmk}
  \begin{enumerate}
  \item \(\polyex{a}{d}{i} \in R[x]^\times \iff a_1, \ldots, a_d\) are nilpotent.
  \item Power Series Ring \(R[[x]]\) is defined the same was a polynomial ring, but we remove the requirement that all
    but a finite amount of terms are zero.
  \end{enumerate}
\end{rmk}

\begin{prop}
  Let \(\varphi \colon R \to S\) be a ring homomorphism, and a choice for where \(x \mapsto a\) there exists a unique
  \begin{figure}[h] \centering
    \begin{tikzcd}
      R \arrow[r,"\varphi"] \arrow[rd,hook]& S \\
      & R[x] \arrow[u,dashed,"\exists! \tilde{\varphi}_a"]
    \end{tikzcd}
  \end{figure}
\end{prop}

\end{document}
%%% Local Variables:
%%% mode: latex
%%% TeX-master: "master.tex"
%%% End:
