\documentclass[master.tex]{subfiles}
\author{Ankit Jaiswal}
\newcommand{\foo}{foobar}
\newcommand{\jc}{John \foo Cena}
\newcommand{\name}[2]{My first name is #1 and second name is #2}
\newcommand{\withDefault}[2][books]{my friends are #1 and #2}

\newtheorem{defn*}{Definition}[section]
\newtheorem{theorem}{Theorem}
%%%%%%%%%%%%%% BEGIN CONTENT: %%%%%%%%%%%%%%

\begin{document}
  \title{How to Structure a LaTeX Document}
  \date{December 2004}
  \subjclass[2010]{03B15 (primary), 20F12, 20F65 (secondary)}
  \begin{abstract}
  Parsing can be fun and this page exactly demostrate that.
  But hey, there is no output without input. That's why I am writing this
  down for it to behave as an abstarct, thus an input :)
  \end{abstract}

  \maketitle

  \section{Testing}
  This follows from \jc the second part of the \textit{remark} above.
  % parsing comments
  \withDefault{Abhijeet}
  Now for some remakrs \% about \foo centralizers. %one more comment.
  \vspace{1cm}
  \begin{rmk*}[1.1.3]
    This is a example of nested environment. \name{Ankit}{Jaiswal} \\
    Also test the line break token. \withDefault[Bhavna]{Abhijeet}
    \begin{enumerate}
    \item\label{pehla} If $A \subgroup G$, \medskip then $A$ is abelian if and only if $A \subset
      C_G(A)$.
    \item Furthermore, if $A \subgroup Z(G)$, then $A \normsubgroup G$. \newline
      This follows using \emph{basic} commutativity arguments, which are as follows:
      \begin{itemize}
        % testing nested list
        \item First One.
        % testing comments.
        \item Second One.
      \end{itemize}
    \end{enumerate}
  \end{rmk*}
  \begin{defn*}[definition of the page]
      For $A \subset G$, we set $N_G(A) := \{g \in G | gAg^{-1} = A\}$ and
      $C_G(A) := \{g \in G | gag^{-1} = a, \forall a \in A\}$.
  \end{defn*}
  Note that $C_G(A) \subset N_G(A)$ and $Z(G) = C_G(G)$.


  \section{math section}
  \begin{equation}
  if x = 2 - x then 2x = 2, hence x = 1
  \end{equation}

  {\tiny\begin{theorem}
  If the \emph{tiny} is followed immediately by begin this fails.
  \end{theorem}
  }

  \subsection{Tyring Floats}
  This can parse graphics as well and label like in \ref{pehla}.
  \begin{figure}[h]
    \includegraphics[width=8cm]{Plot}
  \end{figure}

  \subsubsection{specific case: Tables}
  Trying tables
  \begin{tabular}{ c c c }
   cell1 & cell2 & cell3 \\
   cell4 & cell5 & cell6 \\ [1ex]
   cell7 & cell8 & cell9 \\
  \end{tabular}

  Lets check some code blocks. I know this wont be look good. But real document
  do have a lot plain text material which gives a nice visual appeal.

\section{Tyring code blocks}
  \begin{lstlisting}
  void main() {
    printf("Hello, World!\n");
  }
  \end{lstlisting}

  \subsection{specific case of code style}
  This is to test the latex Numbering.
\end{document}
