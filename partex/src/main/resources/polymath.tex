\documentclass{amsart}

\usepackage{amssymb}
\usepackage{amsmath}
\usepackage{amsthm}
\usepackage{amsbsy}
\usepackage{bm}

\usepackage{hyperref}

\hypersetup{
    colorlinks=true,
    linkcolor=blue,
    filecolor=magenta,
    urlcolor=cyan,
}

\usepackage{listings}
\usepackage{color}

\usepackage{enumitem}



\theoremstyle{plain}
\newtheorem{theorem}{Theorem}[section]
\newtheorem{lemma}[theorem]{Lemma}
\newtheorem{corollary}[theorem]{Corollary}
\newtheorem{proposition}[theorem]{Proposition}
\newtheorem*{quotthm}{Theorem}
\newtheorem{question}[theorem]{Question}
\newtheorem{conjecture}[theorem]{Conjecture}



\theoremstyle{definition}
\newtheorem{definition}[theorem]{Definition}

\theoremstyle{remark}
\newtheorem{remark}[theorem]{Remark}
\newtheorem*{acknowledgements}{Acknowledgements}

\newcommand{\Z}{\mathbb{Z}}
\newcommand{\R}{\mathbb{R}}
\newcommand{\C}{\mathbb{C}}
\newcommand{\N}{\mathbb{N}}
\newcommand{\Q}{\mathbb{Q}}
\newcommand{\F}{\langle \alpha, \beta \rangle}

\newcommand{\co}{\colon\thinspace}
\newcommand{\U}{\mathcal{U}}
\newcommand{\g}{\mathfrak{g}}
\renewcommand{\t}{\mathfrak{t}}
\renewcommand{\c}{\mathfrak{c}}
\newcommand{\h}{\mathfrak{h}}
\newcommand{\e}{\mathfrak{e}}
\newcommand{\B}{\mathcal{B}}


\begin{document}

\title[Intertwined computer \& human proofs]{Homogeneous length functions on Groups: \\ Intertwined computer \& human proofs}

\author{Siddhartha Gadgil}

\address{	Department of Mathematics,\\
		Indian Institute of Science,\\
		Bangalore 560012, India}

\email{gadgil@math.iisc.ernet.in}

\subjclass[2010]{03B15 (primary), 20F12, 20F65 (secondary)}

\date{\today}

\begin{abstract}
We describe  a case of an interplay between human and computer proving which played a role in the discovery of an interesting mathematical result~\cite{polymath}. The unusual feature of the use of computers here was that a computer generated but human readable proof was read, understood, generalized and abstracted by mathematicians to obtain the key lemma in a significant mathematical result.
\end{abstract}

\maketitle

\section{Introduction}

Computers have come to play many roles in mathematical proofs. Computer experimentation is commonly used to make conjectures and computer algebra systems are used for sophisticated calculations. Components of proofs of import results have also been provided by computers. Such rigorous computer proofs often generate independently verifiable \emph{proof certificates}. Proof assistants have been used to formalize proofs, including some very complex ones.

Here we describe a case different from these -- where a computer generated but \href{https://github.com/siddhartha-gadgil/Superficial/wiki/A-commutator-bound}{human readable} proof  was read, understood, generalized and abstracted by mathematicians to obtain the key lemma in a significant mathematical result. So far as we know this is the only such instance so far.

The result whose discovery involved a computer-generated proof concerned the existence of so called \emph{homogeneous length functions} $l: \F \to \R$ on the free groups on two generators (we give precise statements in Section~\ref{S:Homogeneous}). This was asked by Terrence Tao on his blog (Apoorva Khare had asked Tao this question) and was answered  in six days in a collaboration that became PolyMath~14\footnote{\textbf{Participants:} T. Fritz, S. Gadgil, A. Khare, P. Nielsen, L. Silberman, T. Tao.}.

Our computer generated proof of a bound on $l([\alpha, \beta])$ for homogeneous length functions $l$ played a role in the discovery of the proof of the main result. The computer proof was posted in a human readable form -- this was studied, generalized and abstracted to prove the key lemma.

We feel that such proofs intertwining human and computer proving are significant. Mathematical proofs not only establish the truth of statements but also form the basis of further mathematical development. Indeed, a (human) proof of a significant mathematical result inevitably results in further mathematical discoveries. However computer generated proofs, even of important mathematical results, are generally not very useful in this way.

Underlying the computer generated, human readable proofs were \emph{type theory} based \emph{domain specific foundations}. Namely, we defined a \emph{class} whose \emph{objects} were proofs of statements of a specific nature. Here the terms class and object can be understood both in the sense of the words in common English usage and in the sense of object-oriented programming. We also implemented algorithms to generate such proof objects. Further, it was straightforward to transform a proof object to a human readable form.

Conceptually our approach was based on the type theory underlying \emph{homotopy type theory}~\cite{hott}, with statements represented by types belonging to an indexed inductive type family, and proofs terms having these types.  We see more details of the type theory and the \emph{scala} implementation in Section~\ref{S:DSF}. We discuss the algorithms used to find (proofs of) bounds in Section~\ref{S:noncross}. Finally we discuss the role of foundations and limitations and extensions of our approach in Section~\ref{S:Conclusions}.

This note is intended to be complementary to~\cite{polymath}, to which we refer for proofs, motivation for the questions, and other details of the mathematics.

\section{Homogeneous length functions and the Internal repetition trick}\label{S:Homogeneous}

To state the main question, we need some definitions.
\begin{definition}
	A \emph{pseudo-length function} on a group $G$ is a function $l: G \to [0, \infty)$ such that
	\begin{itemize}
		\item $l(e) = 0$.
		\item $l(g^{-1}) = l(g)$ for all $g \in G$.
		\item $l(gh) \leq l(g) + l(h)$ for all $g,h\in G$.
	\end{itemize}
\end{definition}

\begin{definition}[Conjugacy invariance]
	A pseudo-length function $l$ on a group $G$ is said to be \emph{conjugacy invariant} if $l(ghg^{-1}) = l(h)$ for all $g, h\in G$.
\end{definition}

\begin{definition}[Homogeneity]
	A pseudo-length function $l$ on a group $G$ is said to be \emph{homogeneous} if $l(g^n) = nl(g)$ for all $g\in G$, $n \in\Z$.
\end{definition}

\begin{definition}[Positivity]
	A pseudo-length function $l$ on a group $G$ is said to be a \emph{length function} if $l(g) > 0$ for all $g\in G \setminus \{ e \}$.
\end{definition}

Recall that the free group $\F$ on $2$ generators $\alpha$ and $\beta$ is the group whose elements are words in $\alpha$, $\beta$, $\alpha^{-1}$ and $\beta^{-1}$, with the only relations being cancellation of pairs of adjacent letter of the form $\alpha$ and $\alpha^{-1}$ or $\beta$ and $\beta^{-1}$ (for example, $\alpha\beta\beta^{-1}\alpha\beta\alpha^{-1} = \alpha\alpha\beta\alpha^{-1}\neq\alpha\beta$).
Terrence Tao posted on his blog the following question, asked to him by Apporva Khare.

\begin{question}
	Is there a homogeneous, conjugacy-invariant length function $l$ on the free group $\F$ on $2$ generators?
\end{question}

It is natural to view this as asking whether a homogeneous, conjugacy-invariant pseudo-length function $l$ on $G$ can also be positive, hence a length functions. Further, we can normalize to assume that $l(s)\leq 1$ for $s=\alpha, \beta$ (hence, by symmetry, also for $s = \alpha^{-1}, \beta^{-1}$). We shall say $l$ is \emph{normalized} if $l(s)\leq 1$ for $s=\alpha, \beta, \alpha^{-1}, \beta^{-1}$.

After the failure of various constructions (by day $3$), the following conjecture seemed likely.

\begin{conjecture}
	For any homogeneous, conjugacy-invariant pseudo-length function $l$ on $\F$, we have $l([\alpha, \beta]) = 0$, where $[\alpha, \beta] = \alpha\beta\alpha^{-1}\beta^{-1}$.
\end{conjecture}

In particular, this conjecture implies that $l$ cannot be a length function.

Several bounds were obtained on $l([\alpha, \beta])$ from the hypothesis, giving bounds that even went below $1$. However, these methods appeared to stagnate with the best bound above $0.9$.

Using computer-assistance, we obtained and posted
 a \href{https://github.com/siddhartha-gadgil/Superficial/wiki/A-commutator-bound}{human readable} proof showing $l([\alpha, \beta]) \leq 0.82$.
 An extract of this proof is as follows\footnote{see \href{https://github.com/siddhartha-gadgil/Superficial/wiki/A-commutator-bound}{https://github.com/siddhartha-gadgil/Superficial/wiki/A-commutator-bound} for the full proof as originally posted.}. Note that we have used somewhat different notation here -- the generators are $a$ and $b$ and their inverses are denoted $\bar{a}$ and $\bar{b}$. We remark that a fully expanded proof actually had over 2000 lines, but avoiding duplication gave the posted $126$ lines.

{\tiny\begin{itemize}[leftmargin=*]
	\item $|\bar{a}| \leq 1.0$
	\item $|\bar{b}\bar{a}b| \leq 1.0$ using $|\bar{a}| \leq 1.0$
	\item $|\bar{b}| \leq 1.0$
	\item $|a\bar{b}\bar{a}| \leq 1.0$ using $|\bar{b}| \leq 1.0$
	\item $|\bar{a}\bar{b}ab\bar{a}\bar{b}| \leq 2.0$ using $|\bar{a}\bar{b}a| \leq 1.0$ and $|b\bar{a}\bar{b}| \leq 1.0$
	\item ... (119 lines)
	\item $|ab\bar{a}\bar{b}ab\bar{a}\bar{b}ab\bar{a}\bar{b}ab\bar{a}\bar{b}ab\bar{a}\bar{b}ab\bar{a}\bar{b}ab\bar{a}\bar{b}ab\bar{a}\bar{b}ab\bar{a}\bar{b}ab\bar{a}\bar{b}ab\bar{a}\bar{b}ab\bar{a}\bar{b}ab\bar{a}\bar{b}ab\bar{a}\bar{b}ab\bar{a}\bar{b}ab\bar{a}\bar{b}ab\bar{a}\bar{b}| \leq 13.859649122807017$ \\ using $|ab\bar{a}| \leq 1.0$ and \\ $|\bar{b}ab\bar{a}\bar{b}ab\bar{a}\bar{b}ab\bar{a}\bar{b}ab\bar{a}\bar{b}ab\bar{a}\bar{b}ab\bar{a}\bar{b}ab\bar{a}\bar{b}ab\bar{a}\bar{b}ab\bar{a}\bar{b}ab\bar{a}\bar{b}ab\bar{a}\bar{b}ab\bar{a}\bar{b}ab\bar{a}\bar{b}ab\bar{a}\bar{b}ab\bar{a}\bar{b}ab\bar{a}\bar{b}| \leq 12.859649122807017$
	\item $|ab\bar{a}\bar{b}| \leq 0.8152734778121775$ using \\ $|ab\bar{a}\bar{b}ab\bar{a}\bar{b}ab\bar{a}\bar{b}ab\bar{a}\bar{b}ab\bar{a}\bar{b}ab\bar{a}\bar{b}ab\bar{a}\bar{b}ab\bar{a}\bar{b}ab\bar{a}\bar{b}ab\bar{a}\bar{b}ab\bar{a}\bar{b}ab\bar{a}\bar{b}ab\bar{a}\bar{b}ab\bar{a}\bar{b}ab\bar{a}\bar{b}ab\bar{a}\bar{b}ab\bar{a}\bar{b}| \leq 13.859649122807017$ by \\ taking 17th power.
\end{itemize}}

This proof was studied, understood and generalized by Pace Nielsen, who called the method the \emph{internal repetition trick}. After several improvements due to Nielsen and Tobias Fritz, this was abstraced by Terrence Tao as the following lemma.
\begin{lemma}
	Let $x$, $y$, $z$, $w$ in $G$ be such that $x$ is conjugate
to both $wy$ and $zw^{-1}$. Then one has
$$l(x)\leq\frac{l(y) + l(z)}{2}.$$

\end{lemma}

Fritz used this to obtain the key lemma.

\begin{lemma}
	Let $f(m,k)=l(x^m [x,y]^k)$. Then $$f(m,k)\leq \frac{f(m-1,k)+f(m+1,k-1)}{2}.$$
\end{lemma}

A probabilistic argument using this, due to Tao, showed that $l([\alpha, \beta])=0$. This answered the question (the main result proved in~\cite{polymath} is actually stronger than the statement of~\ref{main}).
\begin{theorem}[see ~\cite{polymath}]\label{main}
  For every homogeneous, conjugacy-invariant pseudo-length function $l:\F\to [0, \infty)$, we have $l([\alpha, \beta]) =0$. In particular $l$ is not a length function.
\end{theorem}

\section{Domain Specific Foundations using Type theory}\label{S:DSF}

We now turn to the type-theoretic formulations. For background we refer to~\cite{hott}.

For an element $g\in\F$ and a real number $a\in \R$, let $\B(g)(a)$ be the proposition (i.e., logical statement) that $l(g)\leq a$ for all homogeneous, conjugacy-invariant pseudo-lengths on $\F$ with $l(\alpha),l(\beta)\leq 1$. Recall that we can (and do) regard this as a type under the Curry-Howard correspondence.

Let $S$ be the type given by $S=\{\alpha, \beta, \alpha^{-1}, \beta^{-1}\}$, i.e., the symmetrized set of generators. Let  $\F$ and $\R$  be types corresponding to the free group and reals, respectively. For simplicity we abuse notation and regard elements of $S$ as elements of $\F$ (instead of using an inclusion function).

We model bounds on $l(g)$, for $l$ as above and $g\in\F$, as an indexed inductive type family $\B: \prod_{g\in \F}\prod_{a \in \R} \U$. The assumptions on $l$ are essentially captured in the introduction rules. However, we do not explicitly assume symmetry, instead assuming bounds on lengths for generators as well as their inverses. Further, we only assume invariance under conjugacy by generators and their inverses (this immediately implies conjugacy invariance).

In the following definition, the term $\g$ corresponds to the hypothesis that $l(s)\leq 1$ for $s\in S$.
 The terms $\t$, $\c$, $\h$  correspond to  the triangle inequality, conjugacy invariance, and  homogeneity, respectively. The term $\e$ corresponds to $l(e) =0$.
As we only relate bounds on all lengths satisying the hypothesis with other such bounds, and do not consider specific length functions, some of the statements (such as the triangle inequality) have a somewhat compicated form.

\begin{definition}[Types for bounds]
  The type family $\B: \prod_{g\in \F}\prod_{a \in \R} \U$ is the indexed inductive type family freely generated by
  \begin{itemize}
    \item $\g : \prod_{s: S} \B(s)(1)$
    \item $\t: \prod_{g: \F}\prod_{h: \F}\prod_{a: \R}\prod_{b : \R}\B(g)(a) \to \B(h)(b) \to \B(gh)(a + b)$
    \item $\c : \prod_{s: S}\prod_{g: \F}\prod_{a: \R}\B(g)(a) \to \B(sgs^{-1})(a)$
    \item $\h: \prod_{g: \F}\prod_{n: \Z}\prod_{a: \R}\B(g^n)(a) \to \B(g)(a/n)$
    \item $\e: B(e)(0)$
  \end{itemize}
\end{definition}


Note that any well-formed term $p: B(g)(a)$ is a correct proof of the bound $l(g)\leq a$ under the hypothesis that $l$ is a normalized, homogeneous, conjugacy-invariant pseudo-length on $\F$. This is independent of the act of finding the proof. Further, a proof of this form  is not very different from a standard mathematical proof, and it is easy to see how this translates to a human readable form.

As an example, we write some proofs in this sense in a fully expanded form. The term~\ref{abab} below is a proof that $l([\alpha, \beta]) \leq 2$.
\begin{enumerate}[label=(\alph*)]
  \item $\g(\beta) : \B(\beta)(1)$
  \item $\c(\alpha)(\beta)(1)(\g(\beta)) : \B(\alpha\beta\alpha^{-1})(1)$
  \item\label{abab} $\t(\alpha\beta\alpha^{-1})(\beta^{-1})(1)(1)(\c(\alpha)(\beta)(1)(\g(\beta)))(\g(\beta^{-1})) : \B(\alpha\beta\alpha^{-1}\beta^{-1})(2)$
\end{enumerate}



The inductive definition essentially translates into the \emph{scala} code in Listing~\ref{L:Bound}. The terminology \emph{LinNormBound} is that used in the blog discussions at that time, starting from the original question (the rest of this note uses the terminology of~\cite{polymath}).

\begin{lstlisting}[language=scala, caption=scala code for bounds, label=L:Bound, basicstyle=\footnotesize,   stringstyle=\color{codepurple}]
// class whose objects are bounds on lengths
sealed abstract class LinNormBound(val word: Word, val bound: Double)

// bound on lengths of generators
final case class Gen(n: Int) extends LinNormBound(Word(Vector(n)), 1)


// invariance under conjugacy by generators
final case class ConjGen(n: Int, pf: LinNormBound) extends
                          LinNormBound(n +: pf.word :+ (-n), pf.bound)

// the triangle inequality
final case class Triang(pf1: LinNormBound, pf2: LinNormBound) extends
            LinNormBound(pf1.word ++ pf2.word, pf1.bound + pf2.bound)

// homogeneity
final case class PowerBound(
  base: Word, n: Int, pf: LinNormBound) extends
                                    LinNormBound(base, pf.bound/n){
                                      require(pf.word == base.pow(n))}
// l(e) = 0
case object Empty extends LinNormBound(Word(Vector()), 0)
\end{lstlisting}

Note however the inductive type family $\B$ is only partially captured by scala types, with the rest at object level in scala. Specifically, if code does not use homogeneity and typechecks (in scala) with type of the form \emph{LinNormBound(word, bound)}, the corresponding object is in fact a valid proof. However, we need to load to check correctness of proofs involving homogeneity (for which we have the \emph{require} statement). Further the scala types do not determine $g$ and $a$ in $\B(g)(a)$ so these can only be determined at runtime.

Crucially, any object that \emph{loads} (including passing the require test) is a valid proof. This is independent of how the object was created, whether this was algorithmic, manual, or a combination of these. It is also straightforward to translate such an object to a human readable form\footnote{the posted translation was inductive on the cases and used \emph{unicode} to handle the mathematics}.


\section{Non-crossing matchings and bounds using conjugacy invariance}\label{S:noncross}

We now explain the algorithm used to find proofs of bounds. While this was in a public repository, it was not emphasised during the collaboration and its details played no role in the proof of the main result. We have already emphasised that the computer generated proof, and its correctness, were independent of this algorithm.

By an algorithm to give a proof, we mean an algorithm that takes as input an element $g\in\F$ and has output some bound $B(g)(a)$. Formally, this is a \emph{total} function of type $\F \to \sum_{a\in\R}\B(g)(a)$. We define such functions recursively.

First, observe that if we only used the triangle inequality and the bound on generators, then the best bound we get is the \emph{reduced word length}. Namely, by definition any $g\in\F$ can be expressed as a product $g=l_1l_2\dots l_k$ of generators and their inverses. Further, by performing cancellations if necessary, we can assume that the expression $g=l_1l_2\dots l_k$ is reduced, i.e. does not admit further cancellations (being reduced is equivalent to $l_i\neq l_{i+1}^{-1}\ \forall i, 1\leq i <k$ ). It is well known that the reduced form is unique.

By induction, using the triangle inequality we obtain the bound $l(g)\leq k$. This is in fact sharp if we assume only that $l$ is a normalized pseudo-length.

Further, it is straightforward to translate the inductive proof of the bound $l(g)\leq k$ into a recursive (on word length) definition of a function $\varphi: \F \to \sum_{a\in\R}\B(g)(a)$. Assume all words are reduced. We make the following definition.
\begin{itemize}
  \item For the empty word $e$, define $\varphi(e) = (0, \e): \sum_{a\in\R}\B(e)(a)$, whose type is correct as $\e : \B(e)(0)$.
  \item If $g=sh$ and $\varphi(h) = (a, p)$, define $\varphi(g) = (a + 1, \t(s)(h)(1)(a)(\g(s))(p))$, whose type is correct as $\t(s)(h)(1)(a)(\g(s))(p) : \B(g)(a + 1)$.
\end{itemize}

If we now use the hypothesis of conjugacy invariance, we get in general sharper bounds (and their proofs) recursively. Namely, in the case where $g=sh$, we get a bound which is the minimum of $1$ or more bounds (i.e., we get several bounds of which we choose a minimal one), namely
\begin{itemize}
  \item If we have a proof $p: \B(a)(h)$, i.e., a proof of $l(h)\leq a$, by the triangle inequality as above we get a proof of $l(g) \leq a + 1$, i.e., a term of type $\B(g)(a + 1)$.
  \item For each expression of $h$ as $h_1s^{-1}h_2$ we see that $l(g) = l(sh_1s^{-1}h_2) \leq l(sh_1s^{-1}) + l(h_2) = l(h_1) + l(h_2)$ using conjugacy invariant. Given proofs of bounds  $l(h_1)\leq a_1$ and $l(h_2)\leq a_2$, we get a proof of the bound $l(g)\leq a_1 + a_2$, i.e., from terms of type $\B(h_1)(a_1)$ and $\B(h_2)(a_2)$ we obtain a term of type $\B(g)(a_1+ a_2)$.
\end{itemize}

Interestingly, this is also optimal if we only assume that $l$ is a normalized, conjugacy-invariant pseudo-length (see, for instance, \cite{gadgil}).

Note that the above recursive definition can be modified to use any finite collection of proofs of bounds previously obtained. Indeed, this is important for efficiency, even where we do not use homogeneity, so that we avoid repeating computations (this is called memoization).

In particular, we can choose to use homogeneity for specified elements and powers. This is what was done in generating the proofs, with the elements chosen based on experience obtained from the breakdown of attmepted  constructions (specifically one following~\cite{gadgil}).\footnote{The elements chosen were of the form $\alpha[\alpha, \beta]^k$ and a range of powers was taken with the best bound retained. After seeing the final proof it is clear we should have considered the more general form $\alpha^l[\alpha, \beta]^k$.}

\section{Concluding remarks}\label{S:Conclusions}

There were two limitations in our computer searches. The first (and minor one) was the manual choice of elements and powers for which homogeneity was applied. It would have been straightforward to have replaced this by some learning methods, and this only did not happen due to the sheer speed of the discovery of the proof of Theorem~\ref{main}.

The more serious shortcoming was that only explicit bounds for explicit elements could be represented by objects. Hence neither statements involving quantification nor proofs by induction were implemented. Note that this was only a limitation of the scala code, not the underlying type theory.

Indeed, the type theoretic formulation had no restrictions on quantification, and natural embedded in fully general foundations of mathematics, based on Homotopy Type Theory. Further, the formualtion is a reasonably concise and natural representation of real-world mathematics.

\renewcommand{\footnotesize}{\tiny}

To illustrate that this is not an inherent limitation of the state of the art, we embdedded our domain specific foundations in our implementation of homotopy type theory with symbolic algebra, which is part of the ProvingGround project. Further, we formalized the proof of the internal repetition trick.\footnote{see \href{http://siddhartha-gadgil.github.io/ProvingGround/tuts/internal-repetition-for-length-functions.html}{http://siddhartha-gadgil.github.io/ProvingGround/tuts/internal-repetition-for-length-functions.html}}.

Most mathematics in the real world is rigorous only in a local sense, following established conventions of a field. We feel that the natural way to work with this is to make domain specific foundations codifying these, and then embed these in Homotopy Type Theory, using the modular nature of HoTT so that the embedding is both concise and transparent. We hope that this work has been an example of how this can work.

\begin{thebibliography}{10}


\bibitem{gadgil}
S.~Gadgil, {\em Watson--Crick pairing, the Heisenberg group and Milnor
  invariants}, Journal
  of Mathematical Biology 59(1), 2009.

\bibitem{polymath}
D.H.J. PolyMath, \emph{Homogeneous length functions on Groups}, to appear in Algebra and Number Theory.

\bibitem{hott}
Vladimir Voevodsky et al. \emph{Homotopy type theory: Univalent foundations
of mathematics}. Institute for Advanced Study (Princeton), The Univalent
Foundations Program, 2013.

\end{thebibliography}

\end{document}
